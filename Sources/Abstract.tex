\begin{center}
  \textsc{Declaration}
\end{center}

\noindent

I hereby declare that this paper is my original authorial work, which
I have worked out on my own. All sources, references, and literature
used or excerpted during elaboration of this work are properly cited
and listed in complete reference to the due source.

\hspace*{\fill} Patrick Ondika

\newpage

\begin{center}
  \textsc{Acknowledgements}
\end{center}

\noindent

First and foremost, I'd like to thank Jan Mrázek, for creating the RoFI project, supporting me, and serving as a mentor. His encouragement and willingness to help have been very helpful throughout my entire time in the RoFI project. I'd also like to thank my advisor, Jiřík, for giving me feedback on my work and showing me that you can be successful without abandonning your inner 12-year old. The list goes on to all the other amazing people I met at the faculty; tutors and classmates that, often indirectly, gave me someone to look up to and the motivation to chase after them.

\newpage

\begin{center}
  \textsc{Abstract}
\end{center}
%
\noindent
%
RoFI is a platform of metamorphic robots -- robots consisting of individual modules which each work autonomously and have their own simple joints, but can connect to each other and complete more complex tasks. A natural shape for these robots to take is a robotic arm, which can manipulate with the surrounding objects. What separates RoFI arms from pre-built industrial arms is the total number of joints (degrees of freedom), which rises with each module.

An essential task for robotic arms (manipulators) is the act of moving one object from one place to the other. To successfully accomplish this task, we need to plan a trajectory the manipulator can take to reach the object, while avoiding collisions with potential obstacles. Various algorithms for this problem exist, but the complexity of standard methods scales exponentially with each degree of freedom, making them unusable for RoFI arms.

This thesis aims to design and implement an algorithm for trajectory planning of robotic manipulators with a very high degree of freedom. The thesis goes through the process of designing such an algorithm, explains the individual components, and presents the algorithm as a whole. Finally, the results are evaluated in a simulator within the RoFI environment.

\vfill
Keywords: RoFI, Metamorphic robots, Modular robots, Inverse kinematics, FABRIK, Motion planning, Path planning, Robotic manipulator
